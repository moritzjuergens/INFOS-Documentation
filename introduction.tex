% !TEX root =  master.tex
\chapter*{Einleitung}

IT-Sicherheit bedeutet den Schutz von Informationen und insbesondere deren Verarbeitung. Informationssicherheit soll verhindern, dass Daten und Systeme durch unbefugte Dritte manipuliert werden. Der Sinn dahinter ist, dass sozio-technische Systeme (also Mensch und Technik) innerhalb eines Unternehmens/einer Organisation und deren Daten vor Schäden und Bedrohungen geschützt werden. \\
Wir sprechen nicht nur von Informationen und Daten, sondern auch von physischen Rechenzentren oder Cloud-Diensten. Informationssicherheit wird durch die IT-Schutzziele Verfügbarkeit, Integrität, Vertraulichkeit Authentizität, Zurechenbarkeit, Nicht-Abstreitbarkeit und Verlässlichkeit definiert. IT-Sicherheit wird im Privat- und Unternehmensbereich immer wichtiger. Sicherheitslücken zu erkennen und zu beseitigen ist die Aufgabe der IT-Sicherheit. \cite{hornetsecurity_2020} \\
Die Projektarbeit gliedert sich in drei Phasen, in denen Möglichkeiten aufgezeigt werden, diese Sicherheitslücken zu erkennen und gegebenenfalls Maßnahmen zu ergreifen. Zur Demonstration der Phasen 1 und 3 werden Virtual Machines mit Kali Linux verwendet und der Vorgang mit Hilfe von Screenshots verdeutlicht. 


Den Code für die Projektphase zwei ist in dem GitHub Repository unter\\ https://github.com/moritzjuergens/INFOS-Webserver zu finden.

Die Dokumentation als Gesamtes, die Skripte und die Bilder sind im GitHub Repository https://github.com/moritzjuergens/INFOS-Documentation zu finden. 