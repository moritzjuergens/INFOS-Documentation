% !TEX root =  master.tex
\chapter{Zusammenfassung}

\nocite{*}


\section{Fazit}

Das Bewusstsein für und die Auseinandersetzung mit dem Thema Work-Life-Balance ist mit gut 20 Jahren noch sehr jung. Dennoch hat es sich in dieser kurzen Zeit mit Nachdruck im gesellschaftlichen Diskurs und in der individuellen Priorisierung festgesetzt. Die durchweg positiven Auswirkungen von Work-Life-Balance-Maßnahmen lassen sich betriebswirtschaftlich, gesamtgesellschaftlich (volkswirtschaftlich) und auf der Ebene des Individuums nachweisen. Unternehmen stehen eine ganze Reihe von bewährten Maßnahmen zur Erhöhung der Vereinbarkeit von Privatleben und Erwerbsleben zur Verfügung, die sie nutzen müssen, wenn sie nicht abgehängt werden wollen.

\section{Ausblick}

Weltweit betrachtet gibt es noch erheblichen Nachholbedarf zum Thema Work-Life-Balance. Die Ausprägung hängt jedoch mit dem Entwicklungsstand der jeweiligen Gesellschaften und Ökonomien zusammen und kann sich nur im Rahmen der Veränderungen mit entwickeln.
Auch in Deutschland besteht Potential für weitere Verbesserungen, die durch gesetzliche und gesellschaftliche Rahmenbedingungen gesteuert und gestaltet werden können. Im Zuge der Corona-Pandemie wurde aufgezeigt, mit welchem Tempo sich manche Rahmenbedingungen ändern lassen, hier ist insbesondere auf die Ermöglichung von Remote-Arbeit hinzuweisen. Andere Bestrebungen hingegen, die etwa aus der New-Work-Bewegung kommen, oder auf weitere Reduktionen der Wochenarbeitszeit hinwirken oder gar die Einführung eines bedingungslosen Grundeinkommens fordern, werden voraussichtlich noch mehrere Generationen auf eine Realisierung warten müssen.


